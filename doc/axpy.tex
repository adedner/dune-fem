\documentclass[12pt,a4paper]{article}
\newcommand{\dirac}{ {\rm dirac} }
\newcommand{\integral}{ {\rm integral} }
\begin{document}

A local discrete functional is of the form
$$ \lambda = \sum_i u_i\lambda_i $$
where $(u_i)_i$ is the local dof vector and $(\lambda_i)_i$ is the dual
basis. We introduce addition and other operations on these local
functionals. We use mostly large latin letters to denote general
functionals and small letter to denote functions. Special discrete object
like the above functional will be denoted by greek letters in the
following.

Assume that $F$ is some other functional that can act at least on the
local discrete function space. Given a function $f$ we define a new
functional $f*F$ by the following action:
$$ f*F(g) = F(f*g) $$
Addition of functionals is defined as usual
$$ (F+G)(g) = F(g)+G(g) $$

Now reducing this to the discrete setting we for example want to add a
functional $F$ to the above local discrete functional $\lambda$. For this
it is enough to describe the action on the basis set $(\varphi_i)_i$:
$$ (\lambda+F)(\varphi_i) = \lambda(\varphi_i) + F(\varphi_i)
   = u_i + F(\varphi_i) $$
So we need to add the application of $F$ to $\varphi_i$ to the local
degree of freedom $u_i$. 

Special functonals:
\begin{itemize}
\item $\dirac^0_x(g)=g(x)$: 
  $$ (\lambda + \dirac^0_x)(\varphi_i) = u_i + \varphi_i(x) $$
  Now taking a function $g$ we can define
  $$ (\lambda + g*\dirac^0_x)(\varphi_i) = u_i + \alpha g(x)\varphi_i(x) $$
  This corresponds to the \emph{axpy} method available in Dune-Fem.
\item $\dirac^k_x(g)=\nabla^k g(x)$: this is a generalization of the above.
  With this functional the other available \emph{axpy} methods can be
  defined:
  $$ (\lambda + g*\dirac^0_x + d*\dirac^1_x)(\varphi_i) =
     u_i + g(x)\varphi_i(x) + d(x)\grad\varphi_i(x) $$
  Where $g$ is function into the range space and $d$ and function into the
  jacobian space. The products (like $g\varphi_i$) have to be understood as inner products. 
\item $\integral_Q$: this applied to a function $f$ is supposed to
  approximate the integral over $f$: 
  $$ \integral_Q(f) = \sum_{q\in Q} \omega_q f(x_q) $$
  A weighted integral with a weight function $w$ is then for example given
  by $w*\integral_Q$. Adding this functional to $\lambda$ results in
  $$ (\lambda + w*\integral_Q)(\varphi_i) = 
     u_i + \sum_q \omega_q w(x_q)\varphi_i(x_q) $$ 
  which is implemented in the \emph{axpyQuadrature} methods in Dune-Fem.
\end{itemize}
\end{document}
